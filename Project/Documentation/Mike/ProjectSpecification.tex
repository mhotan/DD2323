%%%%%%%%%%%%%%%%%%%%%%%%%%%%%%%%%%%%%%%%%
% Short Sectioned Assignment
% LaTeX Template
% Version 1.0 (5/5/12)
%
% This template has been downloaded from:
% http://www.LaTeXTemplates.com
%
% Original author:
% Frits Wenneker (http://www.howtotex.com)
%
% License:
% CC BY-NC-SA 3.0 (http://creativecommons.org/licenses/by-nc-sa/3.0/)
%
%%%%%%%%%%%%%%%%%%%%%%%%%%%%%%%%%%%%%%%%%

%----------------------------------------------------------------------------------------
%	PACKAGES AND OTHER DOCUMENT CONFIGURATIONS
%----------------------------------------------------------------------------------------

\documentclass[paper=a4, fontsize=11pt]{scrartcl} % A4 paper and 11pt font size

%\usepackage[T1]{fontenc} % Use 8-bit encoding that has 256 glyphs

\usepackage[english]{babel} % English language/hyphenation
\usepackage{amsmath,amsfonts,amsthm} % Math packages


\usepackage{sectsty} % Allows customizing section commands
%\allsectionsfont{\centering \normalfont\scshape} % Make all sections centered, the default font and small caps

\usepackage{fancyhdr} % Custom headers and footers
\pagestyle{fancyplain} % Makes all pages in the document conform to the custom headers and footers
\fancyhead{} % No page header - if you want one, create it in the same way as the footers below
\fancyfoot[L]{} % Empty left footer
\fancyfoot[C]{} % Empty center footer
\fancyfoot[R]{\thepage} % Page numbering for right footer


%----------------------------------------------------------------------------------------
%	TITLE SECTION
%----------------------------------------------------------------------------------------

\newcommand{\horrule}[1]{\rule{\linewidth}{#1}} % Create horizontal rule command with 1 argument of height

\title{	
\normalfont \normalsize 
\textsc{KTH, School of Computer Science and Communication } \\ [25pt] % Your university, school and/or department name(s)
\horrule{0.5pt} \\[0.4cm] % Thin top horizontal rule
\huge Traffic Flow Simulation for Bike Traffic\\ % The assignment title
\horrule{2pt} \\[0.5cm] % Thick bottom horizontal rule
}

\author{Michael Hotan PN: 870522-T599 SN: 0579 5268\\ Alexandre St-Onge} 

\date{\normalsize\today} % Today's date or a custom date

\begin{document}

\maketitle % Print the title

%----------------------------------------------------------------------------------------
% Introduction
%----------------------------------------------------------------------------------------

% To have just one problem per page, simply put a \clearpage after each problem
\section*{Background}

Creating realistic virtual scene has always been a time consuming task. Not only building placement need to be realist, you also 
need to populate your city with props like stop signs and traffic light, pedestrians, cars, etc. Creating these scene can be quite 
complex if you want to have a coherent environment. Adding bicycles traffic in a scene requires you to think about the path the 
cyclist will take, the density of the traffic and a way to render realist models. 
	
\section*{Problem}

Render a simple "biking" scene with realist lightings, props and a well defined path for the bike. Procedurally generate bike traffic 
into the scene using the navigation framework provided by Unity3D. Use appropriate shading technique to render a realistic cyclist 
model. What kind of traffic density makes the scene realistic.
	
\section*{Implementation}

Unity3D will be used for creating the biking scene. Blender will be used to adapt existing 3D models to our need. ShaderLab and CG 
will be used to write shaders for the cyclist model. C\# scripting in Unity will be used to generate the bike traffic, but also to allow the 
user to change the scene parameters (traffic density, day/night, traffic speed) while the application is running.


\section*{Evaluation}

For this project we will be evaluating the performance (i.e., the program can be runned at a reasonnable framerate on a regular laptop), 
we will also explore how our traffic algorithm could be improved by perceptual studies and what aspect are needed to make a realistic bike 
traffic scene.


\end{document}

